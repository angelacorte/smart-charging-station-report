
%----------------------------------------------------------------------------------------
%	STATO DELL'ARTE
%----------------------------------------------------------------------------------------

\section{Stato dell'arte}

Attualmente, le soluzioni che offrono esperienze simili a quella sviluppata nella nostra app possono essere categorizzate
come applicazioni di navigazione, come ad esempio Google Maps.
Queste applicazioni permettono agli utenti di cercare stazioni di ricarica per veicoli elettrici nel territorio
limitrofo e, conseguentemente, di pianificare un percorso che includa una o più soste per la ricarica.

Un problema che questo sistema può causare è che non si conosce lo stato attuale delle colonnine. Di conseguenza,
potrebbe accadere che si crei un itinerario che includa una stazione di ricarica occupata o non funzionante, causando problemi e ritardi all'utente.

L'idea per lo sviluppo di questo progetto ha tratto ispirazione dall'applicazione MooneyGo (ex MyCicero). Tuttavia,
questa app non fornisce questo tipo di servizio ma consente di pagare parcheggi o ticket per i mezzi pubblici
tramite smartphone, inclusi biciclette e monopattini elettrici, che sono localizzabili attraverso l'applicazione.

Il concetto iniziale del progetto è nato considerando che un servizio di questo tipo potrebbe essere esteso anche
alle stazioni di ricarica per veicoli elettrici. In un contesto urbano di medie o grandi dimensioni, è possibile
che un utente debba spostarsi in auto per svolgere delle commissioni. Pertanto, sarebbe vantaggioso se l'utente
potesse sfruttare al meglio il tempo trascorso durante la sosta, ricaricando il proprio veicolo.


Un altro esempio di applicazione presente sul mercato è quella di un provider energetico come Enel X. Questa app consente agli utenti di localizzare le colonnine di ricarica e di effettuare il pagamento della ricarica direttamente tramite l'applicazione. Tuttavia, presenta alcune limitazioni. Innanzitutto, è specifica del provider Enel X, il che significa che mostra solo le colonnine di ricarica appartenenti a questo provider. Questo può essere un inconveniente per gli utenti che desiderano avere una scelta più ampia di stazioni di ricarica, tenendo conto di fattori come i costi e i percorsi più convenienti per il proprio itinerario.

Inoltre, un'altra sfida con cui gli utenti si possono confrontare è la mancanza di informazioni in tempo reale sullo stato delle stazioni. Nonostante l'app permetta di localizzare le stazioni, gli utenti potrebbero non sapere se una di esse è attualmente occupata o disponibile. Questa mancanza di informazioni aggiornate potrebbe portare a situazioni in cui gli utenti arrivano alla stazione di ricarica solo per scoprire che è occupata, causando disagi e ritardi.

=============================================

Riassumere le soluzioni presenti in letteratura inerenti al problema in esame. Per ciascuna, discutere le principali diversità o affinità rispetto al progetto presentato. Nel caso non siano presenti soluzioni direttamente comparabili a quella presentata descrivere comunque le principali tecniche note per affrontare la tematica trattata.\\

Le soluzioni esposte devono essere corredate degli opportuni riferimenti bibliografici. Nel caso si tratti di soluzioni già operative sul mercato, devono essere indicate le fonti (online) dove poter accedere al servizio o approfondirne i contenuti.\\


Vincoli circa la lunghezza della sezione (escluse didascalie, tabelle, testo nelle immagini, schemi):

\vspace{1cm}
\begin{tabular}{l|rr}
                 & Numero minimo di battute & Numero massimo di battute \\
    \hline
    1 componente & 2000                     & 3000                      \\
    2 componenti & 2500                     & 4500                      \\
    3 componenti & 3000                     & 6000                      \\
    \hline
\end{tabular}


\newpage
