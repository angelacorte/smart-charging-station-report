%----------------------------------------------------------------------------------------
%	TESTING E PERFORMANCE
%----------------------------------------------------------------------------------------

\section{Testing e performance}
%intro + scalatest + server express w/postman

Il testing è un processo fondamentale nello sviluppo del software che mira a verificare se un'applicazione
funziona correttamente e soddisfa i requisiti specificati. Serve a individuare e correggere eventuali
errori o difetti nel codice, garantendo che il software sia affidabile, stabile e conforme alle aspettative degli utenti.

L'importanza di testing e performance engineering risiede nella garanzia della qualità del software.
Il testing assicura che il software sia privo di bug e funzioni correttamente, migliorando la fiducia
degli utenti e riducendo i costi legati a eventuali correzioni post-lancio. Le prestazioni,
d'altra parte, influenzano direttamente l'esperienza utente e la soddisfazione del cliente.
Un'applicazione lenta o instabile può respingere gli utenti.

\subsection{Scalatest}
ScalaTest è un framework di testing per il linguaggio di programmazione Scala.
È stato progettato per consentire una varietà di stili di testing, tra cui il testing unitario,
il testing di integrazione e il testing di accettazione.\\

All'interno del progetto, ScalaTest è stato utilizzato principalmente per il testing
unitario delle varie componenti del software lato Scala. Questo framework ha permesso agli
sviluppatori di scrivere test che verificano il comportamento delle singole unità di codice,
come classi e metodi, in isolamento. In questo modo, è stato possibile identificare e risolvere
eventuali bug o problemi nelle componenti del software in una fase precoce dello sviluppo.

\subsection{Client}
Testare un'applicazione web, specialmente con un framework come Svelte, può essere
impegnativo per diverse ragioni. Queste includono la complessità dei componenti,
l'uso di flussi di dati reattivi, la manipolazione del DOM, l'integrazione con servizi
esterni, la reattività dell'interfaccia utente su dispositivi diversi e il testing end-to-end.
Inoltre, gli aggiornamenti del framework richiedono test aggiuntivi per garantire la compatibilità.
In breve, il testing è cruciale per assicurare l'affidabilità di un'applicazione web, ma può essere
complesso a causa delle sfide uniche presentate dall'ambiente web e dall'architettura reattiva di Svelte.

\subsection{Postman}
Postman è stato utilizzato per effettuare vari test di integrazione, simulando le richieste
HTTP inviate dal client alle web API. Questo strumento ha permesso di verificare il corretto
funzionamento dei servizi REST e di identificare eventuali errori o problemi di comunicazione
tra il client e il server.\\

Sono state testate richieste sia nei confronti del server express che nei confronti del server
Akka HTTP. In particolare, sono state testate le richieste di login, di registrazione,
di visualizzazione delle colonnine e richieste di prenotazione o ricarica.\\



\subsection{Charging Station Domain}
\subsubsection{Unit Testing}
\subsubsection{Integration Testing}

\subsection{User Domain}

Esporre lo stato di funzionamento effettivo del sistema progettato ad elaborato concluso. Per ciascuna delle funzionalità salienti devono essere tabellate e discusse le performance riscontrate mediante opportuni test eseguiti in fase di validazione del progetto.\\

I tempi di esecuzione/comunicazione devono essere accompagnati dalle caratteristiche dell'hardware sul quale è eseguito il software.\\


Vincoli circa la lunghezza della sezione (escluse didascalie, tabelle, testo nelle immagini, schemi):

\vspace{1cm}
\begin{tabular}{l|rr}
                 & Numero minimo di battute & Numero massimo di battute \\
    \hline
    1 componente & 2000                     & 3000                      \\
    2 componenti & 2500                     & 4500                      \\
    3 componenti & 3000                     & 6000                      \\
    \hline
\end{tabular}


\newpage