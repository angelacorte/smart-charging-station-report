
%----------------------------------------------------------------------------------------
%	INTRODUZIONE
%----------------------------------------------------------------------------------------

\section{Introduzione}
La mobilità elettrica sta rapidamente prendendo piede nelle varie nazioni europee, in particolare nelle grandi città,
dove la necessità di ridurre l'inquinamento atmosferico e acustico è sempre più sentita.

Con un aumento costante nella vendita di veicoli elettrici, la necessità di una rete
di stazioni di ricarica efficienti e ben gestite è diventata cruciale. Tuttavia,
il crescente numero di veicoli elettrici nelle strade presenta sfide uniche per la loro gestione.

La disponibilità delle stazioni di ricarica e la loro efficienza possono influenzare
significativamente l'esperienza degli utenti e il successo complessivo della mobilità elettrica.

È in questo contesto che il progetto \textit{"Smart Charging Stations"} prende vita. Si mira a
fornire una soluzione completa e avanzata per la gestione delle stazioni di ricarica, con
l'obiettivo di semplificare il processo di ricarica per gli utenti e promuovere una mobilità sostenibile.
\subsection{Obiettivo del progetto}
L'obiettivo primario del progetto \textit{"Smart Charging Stations"} è migliorare l'esperienza della
mobilità elettrica nelle smart cities attraverso l'implementazione di un sistema di gestione avanzato
per le stazioni di ricarica dei veicoli elettrici. Questa soluzione tecnologica mira a semplificare e
ottimizzare l'utilizzo delle stazioni di ricarica, fornendo agli utenti un accesso facile, rapido ed
efficiente alle risorse di ricarica, contribuendo così alla diffusione su larga scala dei veicoli elettrici
e alla promozione di una mobilità sostenibile.

\subsection{Caratteristiche salienti}
Il progetto \textit{"Smart Charging Stations"} si distingue per diverse caratteristiche salienti.

Dal punto di vista dell'utente, l'applicazione sviluppata permetterà di individuare facilmente le
stazioni di ricarica disponibili all'interno della smart city attraverso una mappa interattiva, permettendo anche di
salvare dei punti di interesse.
Questa funzionalità si basa sull'integrazione di tecnologie di geolocalizzazione per applicativi web.

Un altro elemento chiave è la visualizzazione dello stato delle stazioni di ricarica.
Gli utenti potranno vedere in tempo reale se una stazione è occupata, prenotata o libera, permettendo
agli utenti di pianificare la loro sosta prenotando una stazione in anticipo, eliminando la frustrazione di arrivare a una stazione
solo per trovarla occupata.

Inoltre, l'applicazione permetterà di sbloccare le stazioni di ricarica tramite QR code, garantendo un accesso sicuro e veloce alle risorse di ricarica.

\subsection{Contributo tecnologico-scientifico apportato}
Il contributo maggiore che il progetto \textit{"Smart Charging Stations"} ha apportato
è senza dubbio nel campo dello studio di architetture distribuite per tecnologie per smart
cities.

In particolare, esso ha riguardato la progettazione di un'architettura semi-decentralizzata
e l'ottimizzazione dei flussi di dati all'interno del sistema.\\

Ciò è stato possibile grazie all'impiego di tecnologie per l'implementazione di sistemi
altamente concorrenti, come Scala \cite{scala} e Akka\cite{akka}, grazie alle quali è
stato sviluppato il core decentralizzato del sistema.
Ad esse sono state integrate tecnologie web per lo sviluppo di web app, come JavaScript\cite{javascript}
e Svelte\cite{svelte}, e per il deployment di servizi web come Nodejs\cite{node} e MongoDB\cite{mongo}.\\

In conclusione, il progetto \textit{"Smart Charging Stations"} rappresenta il tentativo di
effettuare un passo avanti verso una mobilità elettrica più efficiente e accessibile
nelle smart cities.\\

Grazie alle tecnologie integrate, all'interfaccia intuitiva e alle funzionalità
avanzate, ci proponiamo di favorire l'adozione su larga scala dei veicoli elettrici,
contribuendo così a un futuro più sostenibile e intelligente.


% =====================================================
%Esporre l’obiettivo del progetto dandone una visione complessiva.\\

%Devono essere illustrate le caratteristiche salienti del progetto; deve essere chiara la distinzione tra le tecnologie usate/assemblate durante lo svolgimento dell'elaborato e il contributo tecnologico/scientifico effettivamente apportato dal gruppo.\\


% Vincoli circa la lunghezza della sezione (escluse didascalie, tabelle, testo nelle immagini, schemi):

% \vspace{1cm}
% \begin{tabular}{l|rr}
%                  & Numero minimo di battute & Numero massimo di battute \\
%     \hline
%     1 componente & 2000                     & 3000                      \\
%     2 componenti & 2500                     & 4500                      \\
%     3 componenti & 3000                     & 6000                      \\
%     \hline
% \end{tabular}


\newpage
