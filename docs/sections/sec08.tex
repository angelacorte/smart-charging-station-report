
%----------------------------------------------------------------------------------------
%	PIANO DI LAVORO
%----------------------------------------------------------------------------------------

\section{Piano di lavoro}
È stata concordata l'adozione di un approccio allo sviluppo ispirato alle
metodologie agili \cite{agile}, con sprint settimanali. Gli sprint settimanali sono stati pianificati in
incontri di sprint planning, durante i quali veniva definito il lavoro da svolgere per la
settimana successiva. Questo lavoro veniva equamente suddiviso tra i membri del team, tenendo
conto delle competenze e delle risorse disponibili.\\


A questi incontri sono state aggiunte retrospettive
per ogni sprint per fare il punto sullo stato di completamento del lavoro pianificato.\\

Inizialmente, è stato condotto uno studio approfondito sulle tecnologie disponibili e sullo
stato attuale delle conoscenze nel settore. Questo studio ha permesso di prendere decisioni
informate sulla scelta delle tecnologie da utilizzare e di adottare le metodologie più avanzate
e adatte al contesto di sviluppo.\\

In seguito, è stato pianificato il lavoro da svolgere per ogni sprint, tenendo conto
delle risorse disponibili e delle priorità.\\


\subsection{Metodologia adottata}
Un elemento chiave di questa metodologia è stata l'aggiunta delle retrospettive alla fine di ogni sprint.
Queste retrospettive hanno permesso al team di fare il punto sullo stato di completamento del lavoro
pianificato e di identificare eventuali aree di miglioramento.\\

Questo ciclo di feedback continuo ha contribuito a ottimizzare il processo di sviluppo e a
garantire una progressiva evoluzione del progetto.\\

L'adozione di un approccio agile con sprint settimanali ha consentito al team di sviluppo
di mantenere un alto grado di flessibilità, di adattarsi rapidamente ai cambiamenti e di
garantire una progressiva evoluzione del progetto, il tutto basato su uno studio approfondito
delle tecnologie e delle metodologie disponibili.

\subsubsection{Sprints}

\begin{itemize}
    \item Sprint 1
          \begin{itemize}

              \item Studio comportamento ideale del sistema ad attori (Cortecchia e Micelli - 1 giorno uomo a testa)
              \item Setup CI/CD (Micelli - 1.5 giorni uomo)
              \item Setup tests (Micelli - 1 giorno uomo)
              \item Implementazione dell'attore inerente alla colonnina di ricarica e relativi test (Cortecchia - 2 giorni uomo a testa)
              \item Implementazione dell'attore inerente alla macchina e relativi test (Micelli - 2 giorni uomo)
          \end{itemize}

    \item Sprint 2
          \begin{itemize}
              \item Studio del framework Svelte (Cortecchia - 1 giorno uomo)
              \item Implementazione di una prima versione della web app (Cortecchia - 2 giorni uomo)
              \item Implementazione dei servizi REST (e relativo studio) lato colonnine (Micelli - 2.5 giorni uomo)
          \end{itemize}

    \item Sprint 3
          \begin{itemize}

              \item Integrazione del servizio di mappe e geolocalizzazione (Cortecchia - 2.5 giorni uomo)
              \item Integrazione del servizio di scanning QR code (Cortecchia - 1 giorni uomo)
              \item Utilizzo di Akka Cluster per i servizi delle colonnine di ricarica (Micelli - 2 giorni uomo)
              \item Refactoring e rimozione degli attori inerenti alle macchine (Micelli - 1 giorno uomo)

          \end{itemize}
    \item Sprint 4
          \begin{itemize}
              \item Implementazione web server express (Cortecchia - 1 giorno uomo)
              \item Configurazione database MongoDB (Cortecchia - 0.5 giorno uomo)
              \item Implementazione dei servizi di autenticazione (Cortecchia - 1 giorno uomo)
              \item Implementazione Akka HTTP per la comunicazione con webapp (Micelli - 3 giorni uomo)
          \end{itemize}
\end{itemize}

\subsection{DevOps}
Per quanto riguarda il processo di sviluppo, il gruppo ha adottato alcune tecniche di DevOps
per migliorare la qualità del software e la produttività del team.\\

La prima scelta fatta è stata la suddivisione del progetto in varie repositories indipendenti.
Ciascuna di queste repository è stata gestita secondo le seguenti modalità:

\begin{itemize}
    \item \textbf{Branching model}: è stato adottato un branching model ispirato al GitFlow \cite{gitflow},
          con un branch main per il codice in produzione e vari feature branch per lo sviluppo di nuove
          funzionalità. Questo approccio ha permesso di mantenere un alto grado di flessibilità e di adattarsi
          rapidamente ai cambiamenti.
    \item \textbf{Pull request}: ogni modifica al codice è stata effettuata tramite pull request.
          Questo approccio ha permesso di proteggere al meglio il main branch da modifiche non testate e
          di sfruttare al meglio la Continuous Integration.
    \item \textbf{Continuous Integration}: è stato configurato un sistema di Continuous Integration
          per ogni repository grazie al meccanismo delle Github Actions. In particolare è stato creato
          un workflow che eseguiva il setup di vari ambienti di esecuzione e che eseguiva i test specificati.
\end{itemize}


% In questa sezione devono essere chiariti i compiti svolti da ciascun candidato nel caso in cui il gruppo abbia più di un componente.\\

% Deve essere inoltre esposto il piano di lavoro adottato. A tal fine, per ogni attività svolta durante la preparazione dell'elaborato (ad esempio: studio di una tecnologia, progettazione di un componente, implementazione di un algoritmo ecc…) deve essere chiarita la collocazione temporale e devono essere indicate le risorse impiegate per svolgerla (giorni/uomo). I candidati possono ricorrere a opportuni diagrammi come quello di Gantt.\\


% Vincoli circa la lunghezza della sezione (escluse didascalie, tabelle, testo nelle immagini, schemi):

% \vspace{1cm}
% \begin{tabular}{l|rr}
%  & Numero minimo di battute & Numero massimo di battute \\
%  \hline
%  1 componente & 1000 & 2000 \\
%  2 componenti & 1500 & 3000 \\
%  3 componenti & 2000 & 4000 \\
%  \hline
% \end{tabular}

\newpage




% \begin{tabular}{l|rr}
%  ID Sprint & Attività & Assegnati & Tempistica (gg uomo)
%  \hline
%  1 & Studio colonnine; visualizzazione stato colonnina & Cortecchia, Micelli & xxx
%  \hline
% \end{tabular}

