
%----------------------------------------------------------------------------------------
%	ANALISI DI DEPLOYMENT SU LARGA SCALA
%----------------------------------------------------------------------------------------

\section{Analisi di deployment su larga scala}

L'applicativo è stato progettato tenendo in considerazione la possibilità di essere eseguito su larga scala.\\

Per quanto riguarda il lato client e il server per l'autenticazione sarebbe possibile eseguire il deployment tramite il servizio Vercel.
Grazie alle funzionalità messe a disposizione dal servizio, è possibile mantenere un deploy continuo dell'applicativo,
in modo da poter aggiornare il codice in produzione in maniera semplice e veloce, tramite semplici push sul main branch di GitHub,
grazie all'integrazione del bot offerto da Vercel.\\

\subsection{CI/CD}

L'integrazione delle pipeline su Github è una mossa strategica che ci permette di
rilasciare in modo continuo nuove versioni della nostra applicazione online, evitando
il fastidio delle operazioni manuali come la configurazione dei server e il deployment
dell'applicazione. Questo approccio semplifica il ciclo di sviluppo, ottimizzando
il flusso di rilascio e garantendo aggiornamenti rapidi e senza interruzioni. \\

In pratica, ci consente di concentrarci sul miglioramento dell'applicazione anziché
sulle complesse operazioni di distribuzione.

\subsection{Database}
Il deployment del database su un cluster di MongoDB Atlas ha svolto un ruolo cruciale nel progetto.
Questa soluzione basata su cloud ha permesso di liberarsi dalla gestione dell'infrastruttura, consentendo
di concentrarsi maggiormente sullo sviluppo dell'applicazione stessa. La scalabilità offerta da MongoDB
Atlas ha garantito prestazioni elevate e adattabilità alle crescenti esigenze dell'applicazione.\\

Inoltre, sono stati sfruttati strumenti avanzati per la sicurezza dei dati e il monitoraggio delle attività.
Complessivamente, questa scelta ha contribuito al successo del progetto, garantendo affidabilità e sicurezza.

% In questa sezione va discussa, eventualmente con l'ausilio di opportuni diagrammi (componenti, deployment), l'evoluzione del progetto presentato immaginando che venga adottato su larga scala. I dettagli qui esposti devono quindi astrarre dalle specifiche dell'elaborato qualora l'implementazione sia stata focalizzata su uno scenario isolato.\\

% A titolo d’esempio, qualora applicabile, devono essere evidenziate le criticità che si potrebbero incontrare e devono essere proposte soluzioni tipiche in contesti di \textit{cloud architecture} per garantire un'adeguata \textit{resilienza}, in termini di \textit{availability} e \textit{scalability} del sistema.\\


% Vincoli circa la lunghezza della sezione (escluse didascalie, tabelle, testo nelle immagini, schemi):

% \vspace{1cm}
% \begin{tabular}{l|rr}
%                  & Numero minimo di battute & Numero massimo di battute \\
%     \hline
%     1 componente & 3000                     & 6000                      \\
%     2 componenti & 4500                     & 9000                      \\
%     3 componenti & 6000                     & 12000                     \\
%     \hline
% \end{tabular}


\newpage
