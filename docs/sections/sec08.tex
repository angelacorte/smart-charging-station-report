
%----------------------------------------------------------------------------------------
%	PIANO DI LAVORO
%----------------------------------------------------------------------------------------

\section{Piano di lavoro - A}
Il gruppo ha concordato sull'adottare un approccio allo sviluppo ispirato alle metodologie agili, con sprint settimanali. In particolare, il gruppo ha tenuto incontri di sprint planning
in cui veniva pianificato il lavoro settimanale sulla base dei requisiti utente, da suddividersi equamente tra i membri. A questi incontri sono state aggiunte retrospettive
per ogni sprint per fare il punto sullo stato di completamento del lavoro pianificato.\\

\subsection{Metodologia adottata - A}
ispirata alle metodologie agili con sprint settimanali

\subsubsection{Sprints - A}


% In questa sezione devono essere chiariti i compiti svolti da ciascun candidato nel caso in cui il gruppo abbia più di un componente.\\

% Deve essere inoltre esposto il piano di lavoro adottato. A tal fine, per ogni attività svolta durante la preparazione dell'elaborato (ad esempio: studio di una tecnologia, progettazione di un componente, implementazione di un algoritmo ecc…) deve essere chiarita la collocazione temporale e devono essere indicate le risorse impiegate per svolgerla (giorni/uomo). I candidati possono ricorrere a opportuni diagrammi come quello di Gantt.\\


% Vincoli circa la lunghezza della sezione (escluse didascalie, tabelle, testo nelle immagini, schemi):

% \vspace{1cm}
% \begin{tabular}{l|rr}
%  & Numero minimo di battute & Numero massimo di battute \\
%  \hline
%  1 componente & 1000 & 2000 \\
%  2 componenti & 1500 & 3000 \\
%  3 componenti & 2000 & 4000 \\
%  \hline
% \end{tabular}

\newpage




% \begin{tabular}{l|rr}
%  ID Sprint & Attività & Assegnati & Tempistica (gg uomo)
%  \hline
%  1 & Studio colonnine; visualizzazione stato colonnina & Cortecchia, Micelli & xxx
%  \hline
% \end{tabular}

