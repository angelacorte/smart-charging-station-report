
%----------------------------------------------------------------------------------------
%	IMPLEMENTAZIONE
%----------------------------------------------------------------------------------------

\section{Implementazione}\label{sec:implementazione}


\subsection{User App - A}

Considerazioni generali sulle particolarità di svelte
\subsubsection{Svelte store}
\subsection{Auth Server}
parlare di express e nodejs e mongodb

\subsection{Cluster}
\subsubsection{Actor Model}
\subsubsection{Akka Cluster}
\subsubsection{Akka HTTP}

\subsection{Tecnologie impiegate - A}
In questa sezione elencheremo le principali tecnologie utilizzate per implementare il sistema e le motivazioni che ci hanno spinto a sceglierle.
...

===================================================================================== \\
Esporre i principali problemi affrontati durante l'effettiva realizzazione delle componenti hardware/software e illustrare le soluzioni implementative adottate. Se l'elaborato ha previsto l'utilizzo di tecnologie già disponibili sul mercato, discuterne brevemente le caratteristiche e motivarne l'adozione rispetto ad altre soluzioni assimilabili.\\

\textbf{NOTA: in questa sezione devono essere riportate esclusivamente le porzioni di codice ritenute particolarmente significative. Il codice sorgente nella sua interezza, opportunamente commentato, deve essere consegnato separatamente dalla relazione in un archivio compresso.}\\


Vincoli circa la lunghezza della sezione (escluse didascalie, tabelle, testo nelle immagini, schemi):

\vspace{1cm}
\begin{tabular}{l|rr}
 & Numero minimo di battute & Numero massimo di battute \\
 \hline
 1 componente & 5000 & 11000 \\
 2 componenti & 8000 & 16000 \\
 3 componenti & 10000 & 21000 \\
 \hline
\end{tabular}


\newpage